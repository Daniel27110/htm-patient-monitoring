\chapter{Conclusiones}

\section{Discusión}

El presente trabajo ha explorado el uso de modelos de búsqueda en subespacios para el análisis de datos fisiológicos provenientes de monitores médicos, con el objetivo de mejorar la detección de anomalías y la interpretación de alertas clínicas. A través del análisis de un conjunto de datos sintéticos y reales, se ha demostrado que este enfoque no solo es viable, sino que también ofrece ventajas significativas en términos de sensibilidad y especificidad en la identificación de patrones anómalos.

El modelo de búsqueda en subespacios ha mostrado resultados prometedores para mejorar el análisis de pacientes a través de datos provenientes de monitores médicos. Al descomponer el conjunto de señales fisiológicas en parejas de variables, se logró capturar interacciones específicas que no son evidentes cuando se tratan todas las señales de forma conjunta, lo cual permitió detectar patrones anómalos de manera más sensible y localizada. Este enfoque demuestra que la exploración en subespacios puede aportar información clínica relevante y complementar las metodologías tradicionales de monitoreo, favoreciendo una interpretación más detallada del estado de cada paciente.

Asimismo, fue posible mejorar considerablemente la interpretabilidad de las alertas reportadas por el modelo anterior, generando resultados accionables para el personal médico. Debido a que cada subespacio está asociado a una pareja de signos vitales concreta, las alertas indican explícitamente qué combinación de variables está presentando un comportamiento anómalo, lo que facilita la toma de decisiones rápidamente orientadas a intervenciones específicas. En consecuencia, el personal de salud puede identificar con mayor claridad las causas subyacentes de cada alerta y priorizar acciones clínicas basadas en la criticidad de cada subespacio, optimizando así la atención al paciente.

Por último, se conservó la resiliencia ante la intermitencia de las señales y la alta precisión y exactitud en la predicción del estado del paciente al comparar los resultados con métricas clínicas bien establecidas. Esto sugiere que el modelo no solo es robusto frente a la variabilidad inherente de los datos fisiológicos, sino que también puede integrarse eficazmente en flujos de trabajo clínicos reales, proporcionando un valor añadido significativo en la monitorización continua de pacientes críticos.

\section{Trabajo futuro}

A partir de lo visto en este trabajo y considerando las limitaciones actuales del enfoque de búsqueda en subespacios, se identifican diversas líneas de investigación que podrían mejorar aún más la capacidad de detección de anomalías y la pertinencia clínica de las alertas generadas:

\paragraph{Asignación de pesos diferenciados a subespacios críticos}
En futuras implementaciones, resulta necesario estudiar cómo asignar distintos pesos a aquellos subespacios que incluyan variables fisiológicas de mayor relevancia clínica, tales como la presión arterial sistólica y diastólica, así como la frecuencia cardíaca. La ponderación diferenciada permitiría que los subespacios que reflejan cambios más significativos en el estado hemodinámico del paciente tengan un impacto proporcionalmente mayor en el cálculo del nivel de alerta global. Para ello, se propone diseñar un mecanismo de ajuste de pesos basado en indicadores clínicos históricos, validado mediante la correlación entre variaciones en estos subespacios y desenlaces relevantes (por ejemplo, episodios de hipotensión o arritmias). De esta forma, se garantizaría que las alertas respondan con mayor sensibilidad ante alteraciones críticas en la presión arterial o ritmos cardiacos, sin sacrificar la contribución de subespacios menos sensibles pero todavía informativos.

\paragraph{Incorporación de información contextual adicional}
Otra línea de trabajo consiste en integrar datos contextuales relacionados con el paciente, como los tratamientos médicos en curso (por ejemplo, dosis de fármacos vasoactivos o sedantes) y diagnósticos previos (historial de enfermedades cardiovasculares o respiratorias). La inclusión de esta información permitiría enriquecer la generación de alertas al facultar al modelo para interpretar ciertas variaciones en los signos vitales como efectos esperados de un tratamiento o, por el contrario, como signos de deterioro en presencia de comorbilidades detectadas con anterioridad. Para implementar este enfoque, se debe diseñar una capa de preprocesamiento que recoja y normalice dicha información, así como ajustar la fórmula de criticidad para que los subespacios se ponderen de acuerdo con el perfil clínico individual de cada paciente. De este modo, la generación de alertas se volvería más personalizada y contextualizada, reduciendo falsos positivos y aumentando el valor predictivo de cada notificación.

\paragraph{Evaluación en distintos grupos etarios dentro de la población pediátrica}
Dado que las respuestas fisiológicas pueden variar de manera significativa entre recién nacidos, lactantes y niños mayores, resulta indispensable evaluar la utilidad del sistema en diferentes grupos etarios dentro del ámbito pediátrico. En futuros trabajos, se plantea realizar estudios comparativos que analicen la precisión y sensibilidad del modelo en subpoblaciones específicas, de tal forma que se identifiquen posibles sesgos o desviaciones en la detección de anomalías según rangos de edad. Esto implicaría recopilar y etiquetar datos de pacientes pediátricos divididos en categorías etarias (por ejemplo, neonatos, lactantes, preescolares y escolares) y, posteriormente, ajustar los parámetros de cada modelo HTM para optimizar su desempeño en cada grupo. Con esta evaluación diferenciada, se podría determinar si es necesario crear modelos adaptados a rangos de edad particulares o si basta con aplicar un único modelo equipado de reglas de normalización específicas.

\paragraph{Análisis del impacto en la carga de trabajo y tiempos de respuesta del personal de salud}
Finalmente, resulta fundamental estudiar cómo la implementación de este sistema influye en la carga de trabajo del personal médico y en los tiempos de respuesta ante eventos críticos. En trabajos futuros se recomienda diseñar un protocolo de evaluación en entorno real—por ejemplo, en una UCIP piloto—que mida métricas tales como el número de alertas diarias, el tiempo transcurrido entre la generación de una alerta y la intervención clínica, y la percepción subjetiva del equipo de enfermería y médicos respecto a la utilidad y facilidad de uso del tablero de control. Mediante encuestas estructuradas y el análisis de registros operativos, se podrá cuantificar si la generación de alertas basadas en subespacios reduce la sobrecarga de alarmas no relevantes y acelera la toma de decisiones en situaciones críticas. Los resultados de este estudio permitirán validar, desde una perspectiva organizacional y operativa, la efectividad global del sistema en entornos hospitalarios reales.
