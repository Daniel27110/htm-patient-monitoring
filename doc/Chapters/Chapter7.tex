\chapter{Conclusiones}

\section{Discusión}

El presente trabajo ha explorado el uso de modelos de búsqueda en subespacios para el análisis de datos fisiológicos provenientes de monitores médicos, con el objetivo de mejorar la detección de anomalías y la interpretación de alertas clínicas. A través del análisis de un conjunto de datos sintéticos y reales, se ha demostrado que este enfoque no solo es viable, sino que también ofrece ventajas significativas en términos de sensibilidad y especificidad en la identificación de patrones anómalos.

El modelo de búsqueda en subespacios ha mostrado resultados prometedores para mejorar el análisis de pacientes a través de datos provenientes de monitores médicos. Al descomponer el conjunto de señales fisiológicas en parejas de variables, se logró capturar interacciones específicas que no son evidentes cuando se tratan todas las señales de forma conjunta, lo cual permitió detectar patrones anómalos de manera más sensible y localizada. Este enfoque demuestra que la exploración en subespacios puede aportar información clínica relevante y complementar las metodologías tradicionales de monitoreo, favoreciendo una interpretación más detallada del estado de cada paciente.

Asimismo, fue posible mejorar considerablemente la interpretabilidad de las alertas reportadas por el modelo anterior, generando resultados accionables para el personal médico. Debido a que cada subespacio está asociado a una pareja de signos vitales concreta, las alertas indican explícitamente qué combinación de variables está presentando un comportamiento anómalo, lo que facilita la toma de decisiones rápidamente orientadas a intervenciones específicas. En consecuencia, el personal de salud puede identificar con mayor claridad las causas subyacentes de cada alerta y priorizar acciones clínicas basadas en la criticidad de cada subespacio, optimizando así la atención al paciente.

Por último, se conservó la resiliencia ante la intermitencia de las señales y la alta precisión y sensibilidad en la predicción del estado del paciente al comparar los resultados con métricas clínicas bien establecidas. Esto sugiere que el modelo no solo es robusto frente a la variabilidad inherente de los datos fisiológicos, sino que también puede integrarse eficazmente en flujos de trabajo clínicos reales, proporcionando un valor añadido significativo en la monitorización continua de pacientes críticos.

\section{Trabajo futuro}

A partir de lo presentado en este trabajo y considerando las limitaciones actuales del enfoque de búsqueda en subespacios, se identifican diversas líneas de investigación que pueden fortalecer la detección de anomalías y la relevancia clínica de las alertas generadas.

\paragraph{Asignación de pesos diferenciados a subespacios críticos}

En futuras implementaciones, seria ser de gran utilidad explorar la posibilidad de asignar diferentes pesos a los subespacios que incluyen variables fisiológicas de mayor relevancia clínica, como la presión arterial sistólica y diastólica, o la frecuencia cardíaca. Por ejemplo, los subespacios que reflejan cambios significativos en el estado hemodinámico del paciente pueden tener un impacto mayor en el cálculo del nivel de alerta global.

Para lograr esto, se propone ajustar la fórmula de criticidad para que cada subespacio contribuya de manera diferenciada al nivel de alerta general. Esto implicaría definir un conjunto de reglas clínicas que ponderen los subespacios según su importancia en el contexto clínico del paciente, permitiendo así una priorización más efectiva de las alertas.

\paragraph{Incorporación de información contextual adicional}

Otra línea de trabajo consiste en integrar datos contextuales del paciente adicionales, como tratamientos médicos en curso y diagnósticos previos (historial de enfermedades cardiovasculares o respiratorias). Al incluir esta información, el modelo puede interpretar ciertas variaciones en los signos vitales como efectos esperados de un tratamiento, o bien, como señales de deterioro en presencia de condiciones preexistentes.

Para implementar este enfoque, se requiere diseñar una capa de preprocesamiento que recoja y normalice la información contextual. Además, es necesario ajustar la fórmula de criticidad para que los subespacios se ponderen según el perfil clínico individual de cada paciente. Así, la generación de alertas se vuelve más personalizada y contextualizada, lo que contribuye a reducir falsos positivos y aumentar el valor predictivo de cada notificación.

\paragraph{Evaluación en distintos grupos etarios dentro de la población pediátrica}

Las respuestas fisiológicas varían considerablemente entre recién nacidos, lactantes y niños mayores. Por ello, es fundamental evaluar la utilidad del sistema en diferentes grupos etarios dentro del ámbito pediátrico. Se plantea realizar estudios comparativos que analicen la precisión y sensibilidad del modelo en subpoblaciones específicas, identificando posibles sesgos o desviaciones en la detección de anomalías según el rango de edad.

\paragraph{Análisis del impacto en la carga de trabajo y tiempos de respuesta del personal de salud}

Finalmente, es esencial analizar cómo la implementación de este sistema influye en la carga de trabajo del personal médico y en los tiempos de respuesta ante eventos críticos. Se recomienda diseñar un protocolo de evaluación en un entorno real que mida métricas como el número de alertas diarias, el tiempo entre la generación de una alerta y la intervención clínica, y la percepción del equipo de enfermería y médicos sobre la utilidad y facilidad de uso del tablero de control.

Para llevar a cabo este análisis, se sugiere establecer indicadores clave de rendimiento (KPI) que permitan cuantificar el impacto del sistema en la práctica clínica diaria. Esto no solo ayudará a validar la efectividad del modelo, sino que también proporcionará información valiosa para futuras mejoras y adaptaciones del sistema a las necesidades reales del personal de salud.

\section{Conclusiones finales}

En conclusión, el uso de modelos de búsqueda en subespacios para el análisis de datos fisiológicos ha demostrado ser una herramienta valiosa para la detección de anomalías en pacientes pediátricos. Este enfoque no solo mejora la precisión y sensibilidad en la identificación de patrones anómalos, sino que también proporciona una mayor interpretabilidad de las alertas generadas, facilitando la toma de decisiones clínicas informadas.

La capacidad de descomponer las señales fisiológicas en subespacios específicos permite una comprensión más detallada del impacto de cada signo vital en el estado del paciente, lo que resulta en alertas más relevantes y accionables. Además, la resiliencia del modelo ante la intermitencia de las señales y lecturas erróneas asegura su aplicabilidad en entornos clínicos reales, donde la variabilidad de los datos es una constante.

Este trabajo sienta las bases para futuras investigaciones en el campo de la monitorización de pacientes críticos, destacando la importancia de integrar enfoques innovadores que mejoren la atención y seguridad del paciente. Las líneas de trabajo propuestas abren nuevas oportunidades para optimizar aún más la detección de anomalías y la relevancia clínica de las alertas, contribuyendo así a una atención médica más efectiva y personalizada.