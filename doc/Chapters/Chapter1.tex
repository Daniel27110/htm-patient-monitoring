\chapter{Introducción}

En las Unidades de Cuidados Intensivos Pediátricos (UCIP), los pacientes en estado crítico enfrentan condiciones médicas que requieren una atención especializada y constante. Debido a la gravedad de su situación, estos pacientes experimentan variaciones significativas en sus signos vitales, las cuales pueden no ajustarse a los rangos normales para individuos saludables de su misma edad. Además, la complejidad de sus condiciones exige un monitoreo continuo y un análisis detallado por parte de los profesionales de la salud pediátrica, con el objetivo de estabilizar sus condiciones fisiológicas y optimizar sus posibilidades de recuperación.

Uno de los principales desafíos en la atención de pacientes en la UCIP es la capacidad del personal médico para interpretar con precisión los datos que proporcionan los sistemas de monitoreo, los cuales se encargan de medir diversos signos vitales de cada paciente. Esta dificultad, además, se ve acentuada por la desproporción entre el número de pacientes y el personal disponible, lo que limita el tiempo que se puede dedicar a la observación continua y al análisis detallado de cada caso. En este contexto, los sistemas de monitoreo registran múltiples variables fisiológicas esenciales, como la frecuencia cardíaca, la saturación de oxígeno en sangre y la presión arterial, proporcionando información clave para evaluar el estado del paciente \parencite{Areia2021}. No obstante, la gran cantidad de datos generados por cada sistema y la necesidad de comprender cómo estas variables se relacionan entre sí pueden dificultar la identificación temprana de complicaciones, lo cual aumenta el riesgo de que ciertos cambios críticos pasen desapercibidos. Cabe destacar que esta situación es especialmente relevante en pediatría, donde los rangos de normalidad varían fuertemente dependiendo de la edad y la condición clínica de cada paciente \parencite{Leyton2020}.

Ante esta dificultad, una solución efectiva debe facilitar una interpretación intuitiva y oportuna de los parámetros críticos que permiten evaluar el estado del paciente en la UCIP. Para ello, se requiere un enfoque que simplifique la integración de la información y que, al mismo tiempo, oriente al equipo clínico hacia una toma de decisiones basada en indicadores claros y de fácil interpretación. Además, esta solución, en su concepción, debe favorecer la detección temprana de variaciones preocupantes en los signos vitales, de modo que los profesionales de la salud puedan actuar con mayor rapidez y precisión para prevenir complicaciones.

Para abordar estos desafíos, este trabajo propone la integración de los datos generados por los sistemas de monitoreo en una plataforma capaz de analizar múltiples señales fisiológicas. Con este propósito, se emplea un enfoque de búsqueda en subespacios (subspace search), el cual permite examinar de forma sistemática la relación entre diferentes pares de variables fisiológicas, identificando combinaciones particularmente relevantes para la detección de cambios críticos en el estado del paciente. Este análisis se realizará utilizando el algoritmo de \textit{Hierarchical Temporal Memory} (HTM), el cual está diseñado para identificar patrones en series de datos y detectar comportamientos inusuales o anómalos a medida que evolucionan con el tiempo. A través de esta metodología, el sistema busca ofrecer una evaluación más contextualizada del estado clínico, superando las limitaciones de los enfoques tradicionales que se basan únicamente en comparar los signos vitales con rangos de normalidad predefinidos. Para validar su efectividad, la solución será desarrollada con el apoyo del Hospital Militar Central de Bogotá, con el fin de demostrar que este acercamiento permite la generación de alertas clínicas tempranas ante posibles eventos adversos, promoviendo así intervenciones más oportunas y proactivas por parte del equipo médico en las Unidades de Cuidados Intensivos Pediátricos.

Los resultados obtenidos muestran que el modelo basado en búsqueda en subespacios ofrece un enfoque prometedor para mejorar el análisis del estado clínico de los pacientes a partir de los datos generados por los sistemas de monitoreo. En particular, se logra una mejora considerable en la interpretabilidad de las alertas generadas, en comparación con los modelos predictivos previamente desarrollados para el Hospital Militar Central. Las alertas emitidas por el nuevo sistema son más dicientes y accionables, permitiendo identificar claramente el origen específico de cada evento detectado. Además, el modelo presenta tanto una alta precisión y sensibilidad en la predicción del estado del paciente como una notable resiliencia frente a la intermitencia o pérdida parcial de señales fisiológicas, lo que refuerza su utilidad en contextos clínicos reales y dinámicos.