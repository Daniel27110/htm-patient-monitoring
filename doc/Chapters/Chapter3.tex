\chapter{Requerimientos y Especificaciones}

El diseño del sistema propuesto requiere una definición clara de los requerimientos necesarios para guiar su desarrollo, implementación y validación. Estos requerimientos se han clasificado en dos categorías: funcionales y no funcionales. Los primeros hacen referencia a las funcionalidades esperadas del sistema, es decir, qué debe hacer, mientras que los segundos abarcan las propiedades de calidad, restricciones metodológicas y criterios de desempeño que debe cumplir la solución.

\section{Requerimientos funcionales}

Los requerimientos funcionales están directamente relacionados con el procesamiento y análisis de los signos vitales de pacientes en la UCIP. Se dividen en requerimientos de entrada y salida, dada la naturaleza de transformación de datos que realiza el sistema:

\begin{itemize}
  \item \textbf{Entrada:} El sistema debe ser capaz de recibir como entrada un conjunto de datos de monitoreo fisiológico previamente registrados. Estos datos deben incluir múltiples signos vitales por paciente, los cuales pueden presentar intermitencia o dejar de ser , dadas las condiciones dinámicas de atención clínica.

  \item \textbf{Procesamiento:} El sistema debe analizar de forma conjunta las distintas señales fisiológicas, sin necesidad de conocer con anterioridad cuántas ni cuáles están disponibles, y ser capaz de identificar anomalías dentro de su comportamiento. En este contexto, se define una anomalía como una alteración simultánea en dos o más signos vitales respecto a los valores típicos del propio paciente, lo cual puede reflejar un cambio significativo en su condición clínica. Es importante destacar que un paciente puede entrar en un estado anómalo tanto por un deterioro como por una mejora repentina de su estado clínico, ya que ambas situaciones representan un cambio brusco en las mediciones de los signos vitales. La solución propuesta debe ser capaz de distinguir entre estos dos casos, utilizando como referencia los rangos de normalidad esperados para el paciente según su edad.

  \item \textbf{Salida:} El sistema debe generar alertas categorizadas según el nivel de criticidad de las anomalías detectadas y presentar los resultados del análisis mediante gráficos de series temporales, indicadores visualmente el estado del paciente y el origen de las alertas.

\end{itemize}

\section{Requerimientos no funcionales}

Desde el punto de vista de calidad y desempeño, el sistema debe garantizar alta confiabilidad y disponibilidad, con tiempos de respuesta mínimos ante la detección de eventos críticos. En términos de seguridad y privacidad, el sistema debe cumplir con las normativas de protección de datos y seguridad informática aplicables en el ámbito hospitalario, de tal forma que la información sensible de los pacientes esté protegida contra accesos no autorizados.

Otro aspecto clave es la interoperabilidad y usabilidad. La solución debe integrarse sin inconvenientes con los sistemas de gestión hospitalaria existentes, asegurando la compatibilidad con los dispositivos médicos en uso. De igual forma, la interfaz desarrollada debe ser intuitiva y de fácil aprendizaje, minimizando la curva de adaptación para el personal de salud.

Finalmente, en cuanto a la metodología de desarrollo, la solución debe construirse siguiendo un enfoque iterativo, que facilite iteraciones rápidas y mejoras continuas basadas en la retroalimentación de los profesionales de la salud.

\medskip

\section{Restricciones}

El desarrollo e implementación del sistema están sujetos a diversas restricciones que deben ser consideradas para garantizar su viabilidad y alineación con el contexto hospitalario.

\begin{itemize}

  \item \textbf{Restricciones de salud y seguridad:} El sistema no debe interferir con la atención directa del paciente ni generar distracciones que puedan comprometer la seguridad del entorno clínico.

  \item \textbf{Restricciones tecnológicas:} La solución debe ser compatible con los dispositivos médicos y sistemas de gestión hospitalaria existentes, evitando la necesidad de reemplazos costosos o complejas integraciones.

\end{itemize}

Las restricciones aquí planteadas definen los límites dentro de los cuales debe desarrollarse la solución y orientan la selección de tecnologías, metodologías y estrategias de implementación. El cumplimiento de estas consideraciones es fundamental para asegurar la viabilidad y efectividad del sistema en el contexto hospitalario de la UCIP.
