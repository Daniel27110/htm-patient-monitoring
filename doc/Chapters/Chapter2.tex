\chapter{Descripción General}

El objetivo principal de este trabajo es desarrollar un sistema que permita monitorear e interpretar anomalías en los signos vitales de pacientes en la Unidad de Cuidados Intensivos Pediátricos (UCIP), utilizando algoritmos de aprendizaje automático. En este contexto, una anomalía se define como un cambio significativo en los patrones normales de comportamiento fisiológico del paciente, el cual puede representar el inicio de un proceso de deterioro clínico y, por tanto, requiere especial atención.

A partir de la detección de estas anomalías, el sistema busca generar alertas dirigidas al personal médico. A diferencia de las anomalías, que son eventos identificados a nivel de procesamiento de datos, las alertas constituyen indicadores explícitos del estado del paciente, diseñados para informar y apoyar al equipo clínico en la evaluación del riesgo de deterioro. Así, el sistema tiene como propósito no solo identificar posibles eventos críticos antes de que se manifiesten, sino también presentar dicha información de forma comprensible y útil, contribuyendo a una respuesta clínica más rápida y fundamentada dentro del entorno dinámico y demandante de una unidad de cuidados intensivos pediátricos.

\section{Objetivos Específicos}

\begin{itemize}
  \item Implementar y adaptar el modelo HTM para identificar desviaciones significativas en las señales fisiológicas de pacientes en la UCIP, explorando diferentes subespacios que analicen la relación entre distintos signos vitales. De esta manera, se busca capturar cambios sutiles o inesperados que puedan anticipar un posible deterioro clínico.

  \item  Desarrollar una interfaz gráfica que integre y presente, en tiempo real, la información procesada por el sistema, incluyendo tanto las anomalías detectadas como las alertas generadas. Así, se facilita la interpretación del estado del paciente por parte del personal médico, permitiendo un monitoreo más efectivo y una toma de decisiones más ágil.

  \item  Evaluar la efectividad del sistema en términos de métricas como el \textit{F-score} y la tasa de falsos positivos, complementando esta evaluación con retroalimentación del personal médico. Esta validación se lleva a cabo con el fin de asegurar la utilidad práctica y la aplicabilidad de la herramienta en contextos reales de cuidados intensivos pediátricos.

\end{itemize}

\section{Antecedentes}

La integración de tecnologías avanzadas en el monitoreo y análisis de datos médicos ha sido un área de creciente interés en la última década. Diversos estudios han demostrado el potencial de las técnicas de análisis de datos y aprendizaje automático para mejorar la gestión de pacientes críticos \parencite{Varnosfaderani2024}.

Investigaciones recientes han resaltado la eficacia de los modelos analíticos basados en aprendizaje automático (ML) para la detección de anomalías en los signos vitales. \textcite{Rush2018} discuten cómo la aplicación de ML a datos fisiológicos continuamente monitoreados tiene un enorme potencial para mejorar la detección de enfermedades, el apoyo a la toma de decisiones clínicas y la eficiencia en el flujo de trabajo. Además, el estudio señala que la integración de ML con datos de registros médicos puede extraer información de gran utilidad a partir de fuentes de datos actualmente sub utilizadas. No obstante, \textcite{Rush2018} advierten que la efectividad del aprendizaje automático está condicionada a la calidad y validez de los datos utilizados. En particular, los datos deficientes pueden llevar a diagnósticos erróneos y a decisiones clínicas incorrectas, subrayando la importancia de garantizar que la información utilizada sea precisa y confiable para evitar posibles errores diagnósticos.

En el contexto de la detección de anomalías en cuidados intensivos, las técnicas de aprendizaje automático no supervisado han demostrado ser altamente eficaces. Un estudio relevante es el de \textcite{Vargas2023}, donde se desarrolló un modelo basado en el análisis de series temporales de datos médicos para identificar anomalías en signos vitales y generar alertas en tiempo real. Según \textcite{Vargas2023}, \enquote*{el modelo ha logrado una tasa de detección superior al 50\% y se centra en anomalías contextuales relacionadas con la evolución de los pacientes} (p. 51). Este trabajo evidencia cómo la aplicación de técnicas avanzadas de análisis de datos puede mejorar significativamente la capacidad de detectar condiciones críticas y alertar al personal médico de manera oportuna.

En síntesis, la evolución en el uso de tecnologías para el análisis de datos médicos y la detección de anomalías ha demostrado resultados prometedores en la mejora de la atención clínica. Los estudios analizados destacan cómo el aprendizaje automático optimiza la identificación de patrones anómalos en signos vitales, fortaleciendo la capacidad de respuesta en entornos críticos. La integración de estas técnicas con sistemas de monitoreo avanzados tiene el potencial de transformar la gestión de pacientes en estado crítico, proporcionando herramientas más precisas y eficaces para respaldar la toma de decisiones clínicas y minimizar errores diagnósticos.

\medskip

\section{Importancia del problema}

El principal problema abordado en este artículo es la dificultad para monitorear y analizar de manera efectiva los signos vitales de los pacientes en las Unidades de Cuidados Intensivos Pediátricos. Estos pacientes se encuentran en condiciones delicadas, donde cualquier variación en sus signos vitales puede ser indicativa de un deterioro crítico. Además, la rapidez y complejidad de estos cambios hacen que el análisis manual resulte una tarea ardua y, en ocasiones, propensa a errores. Cabe destacar que, en contextos donde la proporción entre médicos y pacientes es altamente desfavorable —como sucede en Colombia y otros países de Latinoamérica—, cada profesional de enfermería atiende en promedio a 5,4 pacientes, mientras que cada auxiliar de enfermería atiende a 2,4 \parencite{Arango2015}. Esta desproporción genera una elevada carga asistencial que, a su vez, incrementa el riesgo de errores en la monitorización y la atención, lo que puede resultar en demoras críticas y, en última instancia, en un aumento de las complicaciones.

De hecho, esta elevada carga asistencial está asociada directamente con un incremento en la tasa de mortalidad. Tal como evidenciaron \textcite{Aiken2002}, cada paciente adicional asignado a una enfermera se vincula a un aumento del 7\% en las muertes durante los primeros treinta días posteriores a la admisión. En consecuencia, es imperativo implementar soluciones que optimicen el monitoreo y la atención en las Unidades de Cuidados Intensivos Pediátricos, garantizando una respuesta rápida ante cualquier señal de deterioro y contribuyendo a la reducción de la mortalidad.

Al comparar estas cifras con las de países no latinoamericanos, se observa que algunas naciones de América del Norte y Europa, con sistemas de salud bien desarrollados y economías que asignan mayores recursos al sector sanitario \parencite{oecd2024society}, presentan una proporción enfermera-paciente más favorable. Por ejemplo, en Canadá, un análisis realizado por \textcite{Ariste2019} calcula que la proporción promedio enfermera-paciente es de aproximadamente 1:4 en hospitales generales. Además, según el \textcite{MinistryHealth2024}, las Unidades de Cuidados Intensivos (UCI) mantienen una proporción de 1:1, es decir, una enfermera por paciente las 24 horas del día. De manera similar, en el Reino Unido, las UCI suelen operar con una proporción de 1:1, mientras que en Alemania, la proporción promedio es de 1:2,5 durante el turno diurno y de 1:3,5 en el turno nocturno \parencite{Depasse1998}.

Ante el desafío que representa la elevada carga asistencial y sus consecuencias críticas, se vuelve imperativo explorar nuevas soluciones. En este sentido, se ha evidenciado una tendencia creciente en la ciencia y la ingeniería hacia la aplicación de tecnologías basadas en análisis de datos y aprendizaje automático para mejorar la atención médica \parencite{Davenport2019}. Estas herramientas permiten procesar grandes volúmenes de información en tiempo real, lo que posibilita respuestas más ágiles y precisas ante situaciones críticas en las Unidades de Cuidados Intensivos Pediátricos.

A su vez, resultados recientes en diversas especialidades—como cardiología, neumología y pediatría—respaldados por estudios como el de \textcite{Karalis2024}, sugieren que la implementación de este tipo de sistemas en el entorno de cuidados intensivos pediátricos podría ofrecer una mejora significativa tanto en la precisión diagnóstica como en la capacidad para predicción temprana de eventos críticos. Karalis describe como el uso de algoritmos de aprendizaje automático ha permitido realizar diagnósticos más certeros y diseñar planes de tratamiento personalizados en pediatría crítica. Asimismo, como lo evidencia \textcite{Adegboro2022} en su revisión sistemática sobre el uso de algoritmos de aprendizaje automático para mejorar la atención en cuidados intensivos, cerca del 78\% de los modelos evaluados superaron a los métodos clínicos convencionales, lo cual destaca el notable potencial de estas tecnologías para transformar la atención médica en entornos críticos.

Por otra parte, la aplicación de soluciones tecnológicas en el ámbito de la salud tiene repercusiones que trascienden el entorno clínico. Desde el punto de vista económico, la optimización en la toma de decisiones y la reducción de errores médicos pueden generar un ahorro significativo en los recursos sanitarios. Ambientalmente, la mejora en la eficiencia operativa se traduce en un uso más racional de insumos y una menor generación de desechos asociados a procesos ineficientes. En el plano social, una atención médica más precisa y oportuna mejora la calidad de vida de los pacientes y sus familias, consolidando la confianza en el sistema de salud.

Además, si bien ya se han implementado soluciones tecnológicas en el Hospital Militar de Bogotá, como el uso de dispositivos de monitoreo médico sofisticados, los cuales son capaces de alertar automáticamente al personal médico cuando alguno de los signos vitales del paciente se encuentra fuera del rango normal para su grupo etario (rangos que deben ser ingresados manualmente por una enfermera). No obstante, estos dispositivos presentan limitaciones importantes en el contexto específico de la Unidad de Cuidados Intensivos Pediátricos. La razón principal es que los pacientes en esta unidad ya se encuentran en estado crítico, por lo que es común que uno o más de sus signos vitales estén fuera del rango convencional. Esto genera un dilema operativo: o bien el dispositivo emite constantes alarmas que resultan molestas para los pacientes y el personal médico, o bien obliga a las enfermeras a ajustar continuamente los umbrales de alarma según las consideraciones del especialista, quien debe evaluar qué valores se consideran normales en función de la edad y la condición clínica del paciente. Esta situación demuestra que, aunque la tecnología está presente, no está plenamente adaptada a las necesidades reales del entorno de cuidados intensivos pediátricos.