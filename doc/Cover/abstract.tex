\begin{titlepage}
  \centering
  \newpage
  \pagestyle{empty}

  \chapter*{Abstract}
  \addcontentsline{toc}{chapter}{Abstract}
  \vspace{0.8cm}

  {\bf\centering Use of Predictive Models for Anomaly Detection in Vital Signs in Pediatric Intensive Care Units} \\
  \vspace{0.8cm}

  \begin{flushright}
    \AuthorName \\
    University of the Andes \\
    Supervised by María del Pilar Villamil Giraldo, Ph.D
  \end{flushright}

  \vspace{0.5cm}

  \justifying The effective monitoring of critically ill pediatric patients in intensive care units (ICUs) is a labor-intensive process that demands meticulous analysis to assess each patient's condition. This work explores the use of machine learning models for predictive analytics to identify early warning signs of clinical deterioration. By integrating physiological data from patient monitoring devices, the proposed system aims to support medical decision-making by generating alerts that indicate potential patient instability. The methodology involves data acquisition, preprocessing, model development, and validation in a clinical setting. This approach highlights the potential of predictive analytics to improve patient outcomes by enabling proactive medical interventions.

  \begin{flushleft}
    \vspace{0.5cm}
    \textbf{Keywords:} Predictive analytics, anomaly detection, patient monitoring, machine learning, clinical deterioration, early warning signs,  proactive intervention.

  \end{flushleft}

\end{titlepage}
