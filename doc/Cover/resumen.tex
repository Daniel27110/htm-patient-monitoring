\begin{titlepage}
  \stepcounter{page}
  \centering
  \newpage
  \pagestyle{empty}

  \chapter*{Resumen}
  \addcontentsline{toc}{chapter}{Resumen}
  \vspace{0.5cm}

  {\bf\centering \ThesisTitle} \\
  \vspace{0.8cm}

  \begin{flushright}
    \AuthorName \\
    Universidad de Los Andes \\
    Dirigida por María del Pilar Villamil Giraldo, Ph.D
  \end{flushright}

  \vspace{0.8cm}

  \justifying El monitoreo efectivo de pacientes pediátricos en estado crítico en unidades de cuidados intensivos (UCIs) es un proceso que requiere una gran carga de trabajo y un análisis meticuloso para evaluar la condición de cada paciente. Este trabajo explora el uso de modelos de aprendizaje automático para la analítica predictiva con el fin de identificar signos de alerta temprana de deterioro clínico. Al integrar datos en tiempo real provenientes de dispositivos de monitoreo de pacientes, el sistema propuesto busca apoyar la toma de decisiones médicas mediante la generación de alertas indicativas de una posible inestabilidad fisiológica. La metodología incluye adquisición y preprocesamiento de datos, desarrollo del modelo y validación en un entorno clínico. Los resultados demuestran el potencial de la analítica predictiva para mejorar los desenlaces clínicos al permitir intervenciones médicas proactivas.

  \begin{flushleft}
    \vspace{0.5cm}
    \textbf{Palabras clave:} Analítica predictiva, detección de anomalías, monitoreo de pacientes, aprendizaje automático, deterioro clínico, signos de alerta temprana, intervención proactiva.

  \end{flushleft}

\end{titlepage}
